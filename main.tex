\documentclass{article}
\usepackage[utf8]{inputenc}

\usepackage{amsmath, amssymb, amsthm, graphicx, epsfig, fancyhdr, enumitem, breqn, bm, geometry, lastpage, titling}
\usepackage[colorinlistoftodos]{todonotes}
\usepackage[utf8]{inputenc}
\usepackage[mathscr]{euscript}
 \geometry{
 a4paper,
 total={170mm,227mm},
 left=20mm,
 top=30mm,
 }

\title{Parameter Values} 
\author{Matthew Moreno}

\setlength{\headheight}{28pt}
\setlength\parindent{0pt}
\theoremstyle{definition}
\newtheorem{definition}{Definition}
\newtheorem{lemma}{Lemma}

\usepackage{xspace}
\usepackage{mathtools}
\newcommand\numberthis{\addtocounter{equation}{1}\tag{\theequation}}
\DeclarePairedDelimiter\abs{\lvert}{\rvert}%
\DeclarePairedDelimiter\norm{\lVert}{\rVert}%

\begin{document}
    

EVENT PARAMETERS
\begin{itemize}
	\item  specify random motion component
	\begin{itemize}
    	\item sigma\_rm specifies the standard deviation of ant turn angles 
      	\[ \texttt{sigma\_rm} = 1 \]
        source: \cite{khuong_how_2013} pg 12, worked out in \texttt{turn_deviation.nb}
      
      	\item \texttt{beta\_rm} affects rate (event/distance) at which ant undergoes reorientation events (its mean free path), \texttt{slope\_freepath} causes lengthening of free path as ant travels in alignment with gradient on steeper slope -- WARNING: this will get weird for inclines much past $\pi/3$...
      	\begin{align*}
        	\text{free path} 
            &= \texttt{beta\_rm} + \operatorname{abs}(\vec{s} \cdot \hat{\vec{v}}) \texttt{slope\_freepath}  \\
            \texttt{beta\_rm} = 1  \text{ cm}\\
            \texttt{slope\_freepath} 
            &= \frac{0.6}{\tan(\pi/3)} \text{ cm}	 \\		
            &= 0.3464  \text{ cm}
        \end{align*}
                source: \cite{khuong_how_2013} pg 7

      	\item turning due to orientation relative to incline--straight or turn
        \begin{align*}
			t =
			\begin{cases} 
      			0 & \bm{U}_1 < f(\phi) \\
      			1 & \text{otherwise}
   			\end{cases}
		\end{align*}
        \begin{align*}
        	f(\phi) 
            &= b_1 + a_1\cos(2\phi)(c_1 - \hat{\vec{s}} \cdot \hat{\vec{v}}) \\
            &= b_1 + a_1[\cos^2(\phi)-\sin^2(\phi)](c_1 - \hat{\vec{v}} \cdot \hat{\vec{s}}) \\
            f(\vec{s}, \vec{v}) 
            &= b_1 + a_1\Big[\frac{(\vec{s} \cdot \vec{v})^2}{(\vec{v}\cdot\vec{v}) (\vec{s} \cdot \vec{s})}-\frac{(s_xv_y-s_yv_x)^2}{(\vec{v}\cdot\vec{v}) (\vec{s} \cdot \vec{s})}\Big](c_1 - \hat{\vec{v}} \cdot \hat{\vec{s}}) \\
            &= b_1 + a_1\Big[\frac{(\vec{s} \cdot \vec{v})^2 - (s_xv_y-s_yv_x)^2}{(\vec{v}\cdot\vec{v}) (\vec{s} \cdot \vec{s})}\Big]\Big(c_1 - \frac{\vec{s} \cdot \vec{v}}{\norm{\vec{v}}\norm{\vec{s}}}\Big) 
        \end{align*}
        \begin{align*}
        \texttt{a\_1} = 0.0500, \texttt{b\_1} = 0.1625, \texttt{c\_1} = 1.5
        \end{align*}
        source: \cite{khuong_how_2013} (based on visual approximations from data in Figure 7), worked out parameters in \texttt{straight_probs.py}, worked out vector math/simplification in this document and in \texttt{sin_vecs.py}, verified result give here in \texttt{confirm_vec_math.py}
        
        \item turning due to orientation relative to incline--left or right
		\begin{align*}
			s =
			\begin{cases} 
      			1 & \bm{U}_2 <  g(\phi) \\
      			-1 & \text{otherwise}
   			\end{cases}
		\end{align*}
        \begin{align*}
        	g(\phi) 
            &= 1/2 + d_2\frac{(a_2\sin(\phi) - b_2\sin(2 \phi))}{\pi} \\
            &= 1/2 + \frac{a_2 d_2 \sin(\phi)}{\pi} - \frac{2 b_2 d_2 \sin(\phi) \cos(\phi)}{\pi}
 \\
            g(\vec{s},\vec{v}) 
            &= 1/2 + m_2 \frac{s_xv_y-s_yv_x}{\norm{\vec{v}}\norm{\vec{s}}} - n_2 \frac{s_xv_y-s_yv_x}{\norm{\vec{v}}\norm{\vec{s}}} \frac{\vec{s} \cdot \vec{v}}{\norm{\vec{v}}\norm{\vec{s}}} \\
            &= 1/2 + m_2 \frac{s_xv_y-s_yv_x}{\norm{\vec{v}}\norm{\vec{s}}} - n_2 \frac{(\vec{s} \cdot \vec{v})(s_xv_y-s_yv_x)}{(\vec{v} \cdot \vec{v})(\vec{s}\cdot\vec{s})}
        \end{align*}
        \begin{align*}
       		\texttt{a\_2} = 0.8139, \texttt{b\_2} = 0.5482, \texttt{d\_2} = 0.9154 \\
            \texttt{m\_2} = \frac{\texttt{a\_2} \times \texttt{d\_2}}{\pi} = 0.2372 \\
            \texttt{n_2} = \frac{2 \times \texttt{b\_2} \times \texttt{d\_2}}{\pi} = 0.3195
        \end{align*}
  		source: \cite{khuong_how_2013} (based on visual approximations from data in Figure 7), worked out parameters in \texttt{turn_probs.py}, worked out vector math/simplification in this document and in \texttt{sin_vecs.py}, verified result give here in \texttt{confirm_vec_math.py}

      	\item the distance from food/nest at which the ant switches roles
      	\[ \texttt{dist\_switch\_role\_squared} = 1 \text{ cm}^2; \]
       	source: none (seemed like a convenient and reasonable choice)
	\end{itemize}
    \item pheromone grid and behavior
    \begin{itemize}
     	\item spacing between grid points
      	\[ \texttt{phgrid\_size} = 0.1 \text{ cm} \]

      	\item how far away from current position ant can sense pheromone
      	\[ \texttt{sense\_radius} = 2 \text{ cm}\]
        source: \cite{perna_individual_2012} pg 10
	\end{itemize}
\end{itemize}
DIFFERENTIAL EQUATION PARAMETERS
 \begin{itemize}
%  	\item target running speed of ant
%     \[ \texttt{xi} = 2 \]
%     source: \cite{perna_individual_2012} pg 12 (ants observed going up to 6 cm / sec... this was the average observed speed)
    
    \item constant power equation
    \[ \texttt{ant\_power} = \norm{v}^2[\texttt{incline\_scale} - cos(\theta)] + \texttt{gravity\_scale} sin(\theta) \norm{v}\]
    matlab \texttt{best\_fit.m}, grapher, \cite{holt_locomotion_2012} $\rightarrow$ $\texttt{power} = 0.8115$, $\texttt{incline\_scale} = 1.2144$, $\texttt{gravity\_scale} = 0.2005$

%     \item affects rate at which ant accelerates under its own power
%     \[ \texttt{nu} = 1 \]
%     source: \cite{ryan_model_2016} pg 1589

    \item $e^{-\text{coeff}} \times \text{dist}$ -- smaller = more attraction
    \[ \texttt{attraction\_to\_food\_1} = 1 \]
    source \cite{ryan_model_2016} pg 12

    \item $\text{coeff} \times e^{(-\text{dist})}$ -- larger = more attraction \\
    according to Simon half life of pheromone is 20 minutes \\
    taking into account the diffusion of pheromone, the ``effective'' half life might be much lower than this
    then $\lambda = \ln(1/2)/1 \approx -0.693$
    \[ \texttt{attraction\_to\_food\_2} = 0.693 \]

    \item rate at which ant proceeds towards nest
   	\[ \texttt{attraction\_to\_nest} = 1 \]
    source: \cite{ryan_model_2016} pg 1589

    \item rate at which ant deposits pheromone
    \[ \texttt{pheromone\_rate} = 1 \]

    \item half life of pheromone is about 20 min \\
    $\lambda = \ln(1/2)/1200$, $1200 = 20 \times 60$ sec
    \[ \texttt{evap\_lambda} = -0.0005776 \]
    source: personal communication with Simon
    

    \item how strongly does an ant respond to pheromone
    \[ \texttt{attraction\_to\_pheromone} = 10 \]
    \cite{ryan_model_2016} pg 1589

    \item how long and wide is the ant
    \[ \texttt{ant\_length} = 0.5 \text{ cm} \]
    \[ \texttt{ant\_width} = 0.3 \text{ cm} \]
    source: personal communication with Simon
\end{itemize}

OTHER PARAMATERS
\begin{itemize}
	\item number of ants in the simulation
    
    Ryan's nest$\leftrightarrow$food path is 60 nondimensional units, approximately 200 times the length of an individual ant. An ant has length around 0.5 cm.
    \[
    1 \text{ndu} \frac{200 \text{ants}}{60 \text{ndu}} \frac{1 \text{cm}}{2 \text{ant}} = 1.67 \text{cm}
    \]
    Ryan's arena 100 ndu by 50 ndu, has 400 ants.
    \[
    \frac{400 \text{ant}}{5000 \text{ndu}^2} \Big(\frac{1 \text{ndu}}{1.67\text{cm}}\Big)^2 = 0.02868514468 \text{ant/cm}^2
    \] 
    Our arena is 10 cm by 30 cm.
    \[
    0.02868514468 \text{ ant/cm}^2 \times 300 \text{ cm}^2 = 8.605 \text{ ant}
    \]
    Ryan's nest$\leftrightarrow$food path is 60 nondimensional units $\approx$ 100 cm $\rightarrow$ 4 ants/cm. Our path is 20 (?) cm (if nest/food 5 cm in) $\rightarrow$ we should have 80 ants
\end{itemize}

\nocite{*} % Insert publications even if they are not cited in the poster
\bibliographystyle{plain}
\bibliography{mbi_reu_bib}

\end{document}
